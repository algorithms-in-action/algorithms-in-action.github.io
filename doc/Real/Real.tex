\documentclass[11pt]{article}
\usepackage{url}
\pagestyle{empty}
\textwidth	160mm
\textheight	238mm
\topmargin      -8mm
\oddsidemargin	-2mm
\evensidemargin	-2mm
\setlength{\parskip}{0.7em}
\setlength{\parindent}{0ex}
 \renewcommand{\baselinestretch}{1.04}
\newenvironment{mylist}{\begin{list}{$\bullet$}
  {\setlength\labelwidth{.5em}\setlength\leftmargin{1em}
  \setlength{\itemsep}{-0.4ex}
  \def\makelabel##1{##1\hfill}}}{\end{list}}

\newcommand{\id}[1]{\mathit{#1}}
\newcommand{\slant}{\texttt{\char'134}}        % Backslash
\newcommand{\mycaret}{\char'136}               % Caret
\newcommand{\mytilde}{\char'176}               % Tilde

\begin{document}
\begin{center}
{\LARGE\bf A Specification Language for Algorithms in Action} \\[1ex]
Lee Naish, Linda Stern, and Harald S{\o}ndergaard \\
Draft, \today
\end{center}

The main purpose of this document is to define a language for
specifying how an algorithm is to be displayed in Algorithms in Action
(AIA). Those with expertise in algorithms and teaching can use this
language to communicate requirements to AIA developers.  We have named
the language Real, for {\it Refinement Explanation Animation Language}.
The language shows pseudocode, explanations, and the desired breakdown of
the pseudocode into blocks of pseudocode that can be expanded (refined),
explained, and collapsed by the user.  In this document, we define the
language, make some suggestions for how it is used and give an example
of its use for Heapsort.

An advantage of Real is that it is quite easily readable by both humans
\emph{and} machines.  Consequently, it is possible that specifications
written in Real can be incorporated into AIA code almost \emph{as
is}.  We hope that such a strategy will contribute to both ease
of implementation and longevity of the AIA software, since Real is
relatively self-explanatory and could be understood by future developers
without difficulty.


\section{Language specification}

In Real, pseudocode to be displayed in AIA is written simply as lines
of text.  The language also contains {\it directives}, or instructions,
that are distinguished from lines of pseudocode by starting with a
backslash (\textbackslash).  Because all directives begin with the
backslash character, it is simple to grow the language to include further
directives, if necessary, while maintaining its conceptual integrity.
The current directives are:

\begin{description}

\item{\verb~\Note{~:}
Start of a note for developers etc. A note is not displayed in AIA.

\item{\verb~\Note}~:}
End of a note. All text between the start of the note and the end of the note, 
plus any trailing newline, is ignored for the purposes of display.

\item{\verb~\Overview{~:}
Start of the overview of the algorithm displayed in AIA.

\item{\verb~\Overview}~:}
End of the overview.

\item{\verb~\Code{~:}
Start of a pseudocode code block, that is, a chunk of pseudocode. 
The first line of a block is a label that uniquely refers to that block, 
and is not displayed. 
The label for the top level code block is generally `\texttt{Main}' but
there may be multiple ``entry points'', eg for insertion and searching. 
Other code blocks are refinements, that is expansions, of lines of code 
that appear elsewhere. 
The block label links the block with the line of code it is expanding, 
see {\tt \textbackslash Ref} below.  
 
Subsequent lines of a block, after the identifier, are the pseudocode, 
plus the relevant explanations or other directives.

\item{\verb~\Code}~:}
End of a code block.

\item{\verb~\Ref~:}
A line of code may end in \texttt{\slant}\texttt{Ref},
followed by a label that names another code block.
The  \texttt{\slant}\texttt{Ref} directive shows that the line of pseudocode is 
expandable: the user can perform an action, such as a mouse click, to replace 
the line by the code block of the name specified by 
\texttt{\slant}\texttt{Ref}.
Similarly, the code block specified by \texttt{\slant}\texttt{Ref} can be 
condensed, to show, once again, the original line of pseudocode. 
(See, for example, the line 
{\tt for k <- Index of last non-leaf downto 1 do \textbackslash Ref BHForLoop} 
in the Heapsort example below, where \texttt{\slant}\texttt{Ref} specifies 
that this line can be further refined on user request, and specifies where 
the pseudocode for this expansion is to be found.)

\item{\verb~\Expl{~:}
Start of an explanation. This directive appears after a line of pseudocode 
(usually on the next line), and is a plain language explanation of the 
preceding pseudocode.  
The explanation is shown in the explanation display 
(not directly in the pseudocode) when requested by the user by some mechanism, 
possibly clicking on the line of pseudocode.

\item{\verb~\Expl}~:}
End of an explanation. All text between the start and the
end of the explanation (plus any trailing newline) is not 
included in the displayed pseudocode
(but is accessible {\it via} the mechanism chosen for showing explanations).
 Any trailing newline character is also ignored in the display.

\item{\verb~\In{~:}
Used within code blocks to specify deeper indentation.
The lines of pseudocode between {\verb~\In{~} and {\verb~\In}~} will be 
indented one level more than the current indentation level. 
Leading spaces and tabs on code lines are ignored.

\item{\verb~\In}~:}
End indentation, that is, go back to the level of indentation
that was current before the corresponding 
\texttt{\slant}\verb!In{!. 
Any trailing newlines after indentation directives are ignored for 
the purpose of displaying pseudocode.

\item{\texttt{\slant}~~:}
that is, a backslash followed by a space. This will be displayed
as a space (and so can be used to force extra indentation).

\item{\texttt{\slant}\texttt{\slant}:}
This is displayed as a single backslash in pseudo-code.
\end{description}

A note about whitespace: Unless otherwise specified, whitespace in the 
algorithm specification is for human readability only. 
Blank lines outside of a code block contained between {\verb~\Code{~} 
and {\verb~\Code}~} are not shown in the AIA display.  
Similarly, indentations outside of a segment between {\verb~\In{~} and 
{\verb~\In}~} are not displayed.

Additional signals and identifiers for linking in animation code will
almost certainly be necessary, but are not part of Real. For example,
the current AIA code uses \verb@\B@ followed by a number to indicate a
``bookmark'' used in the animation code.

\section{Tips for writing Real specifications}

For those writing Real specifications, we now provide a few suggestions
based on our experiences with writing pseudo-code for AIA over the
years.

\begin{description}
\item{Forget code efficiency:}
AIA is designed to teach people about algorithms, \emph{not} efficient
coding.  If you are an algorithm expert you probably like neat coding as
well --- try to forget that. No matter how neat your preferred coding is,
the simplest coding that illustrates the algorithm in the clearest way
is better for the goal of AIA.  You can always put extra information in
explanations or the overview mentioning how the algorithm can be coded
more efficiently than the way presented.

\item{Write code as well as pseudo-code:}
Pseudo-code can have bugs, just as code can. By writing code based on
your psuedo-code you can test and debug it. Make sure your code
really has the same structure as the pseudo-code --- if the code can't
easily be structured that way, the pseudo-code probably needs adjustment.
Again, don't worry about efficiency; just concentrate on structure and
correctness.

\item{Include your code in a note:}
This keeps everything relevant in one Real file.  Also, the AIA
developers need to code the algorithm in order to animate it - if you
provide code it makes their job easier.  Code is also unambigous, where
as psuedo-code can potentially be misunderstood.

\item{Include notes on animation:}
The AIA developers need to know how you want the code animated.
It's worthwhile at least having an overview in the Real file, including
any subtle points.  Details will no doubt need further refinement and
discussion with developers but subtleties can be missed.

\item{You can always remove clutter:}
Sometimes Real code can get rather cluttered with notes, explanations,
etc. It's pretty easy to remove them and we have written a very
simple/naive bash script, \texttt{cleanreal} that removes a bunch of
things so the pseudo-code is easier to read as it is being developed.

\item{Try to be consistent:}
Ideally we should have a proper style guide for Real, but currently we
don't. However, you can look at animations and Real code that have been
developed and try to follow the same conventions unless there is a good
reason not to.

\item{Consider common structure of different algorithms:}
Sometimes there are several algorithms with very similar structure. Real
can emphasise this by having the same (or very similar) top level
pseudo-code, which may be beneficial. This sometimes involves compromises
with minor details of some of the algorithms.

\item{Top level pseudo-code should not be trivial:}
Very early in the original development of AIA we had a single
line of pseudo-code at the top level: something like ``\texttt{Sort
array}''. Despite being very elegant, allowing an explanation of what
sorting is and illustration of \emph{what} the algorithm computes in a
single step, it just confused many of the students. The top level needs
to have at least a couple of distinct steps.

\item{You don't have to spell everything out:}
It is reasonable to assume that students using AIA will understand basic
coding and will understand simple concepts before attempting to understand
complex algorithms. If you are writing Real code for something simple like
selection sort it's worth refining a line like ``\texttt{m <- the index
of the minimum element of} \verb@a[i]...a[max]@'' but for more complex
algorithms the extra code in the fully refined version may just be a
distraction from the key points in the algorithm. In theory, the fully
refined version \emph{could} be code that can be executed --- elegant,
but AIA is \emph{not} about coding.  Some students have a ``bottom up''
learning style, where the first thing they do is expand everything,
and if the fully expanded version is not too long it may help these
students especially.

% \item{animation:}

\end{description}

\pagebreak
\section{Example: Heapsort for AIA, Specified in Real}

\begin{verbatim}
\Note{  Real specification of heapsort animation
\Note}
\Overview{
Overview of Heapsort omitted for brevity
\Overview}

\Code{
Main
HeapSort(A, n) // Sort array A[1]..A[n] in ascending order.
    \Expl{  We are not using A[0] (for languages that start array indices at 0).
    \Expl}
\In{
    BuildHeap(A, n) \Ref BuildHeap 
        \Expl{  First reorder the array elements so they form a (max) heap
                (no element is larger than its parent). The root node, A[1],
                is therefore the largest element.  
        \Expl}
    SortHeap(A, n) \Ref SortHeap 
        \Expl{  Convert the heap into a sorted array. The largest element is
                put in the correct position A[n] first and we work backwards 
                from there, putting the next-largest element in its place, 
                etc, shrinking the heap by one element at each step. 
        \Expl}
\In}
\Code}

\Code{
BuildHeap
// build heap
for k <- Index of last non-leaf downto 1 do \Ref BHForLoop 
    \Expl{  We use bottom-up heap creation, to build the heap from the bottom
            up (tree view) and right to left (array view). The leaves are 
            already heaps of size 1, so nothing needs to be done with them. 
            Working backwards through the heap, and starting from the last 
            non-leaf node, we form heaps of up to size 3 (from 2 leaves plus
            their parent k), then 7 (2 heaps of size 3 and their parent k) 
            etc, until the whole array is a single heap. 
    \Expl}
\In{
    DownHeap(A, k, n) \Ref DownHeapk 
        \Expl{  DownHeap is where smaller heaps are combined to form larger
                heaps. The children of node k are already heaps, so we need
                only be concerned about where A[k] fits in. 
        \Expl}
\In}
\Code}

\Code{
BHForLoop
for k <- n/2 downto 1 do 
    \Expl{  Using root index 1, the last non-leaf has index n/2 (rounded down
            to the nearest integer).
    \Expl}
\Code}

\Code{
DownHeapk
// DownHeap(A, k, n)
    \Expl{  DownHeap is where smaller heaps are combined to form larger heaps.
            The children of node k are already heaps, so we need only be 
            concerned about where A[k] fits in. 
    \Expl}
i <- k
    \Expl{  Set index i to the root of the subtree that we are now going to 
            make into a heap. 
    \Expl}
heap <- False // 'heap' is a flag  
while not (IsLeaf(A[i]) or heap) do                       
    \Expl{  Traverse down the heap until the current node A[i] is a leaf. 
            We also terminate the loop if the children of A[i] are in the 
            correct order relative to the parent, since we know that subtrees
            lower down already meet the heap condition. We use the heap flag
            to test the heap condition.  
    \Expl}
\In{        
    j <- IndexOfLargestChild(A, i, n) \Ref IndexOfLargestk 
        \Expl{  Find the larger of the two children of the node.
        \Expl}
    if A[i] >= A[j] then
    \In{
        heap <- True
            \Expl{  The heap condition is satisfied (the root is larger 
                    than both children), so exit from while loop. 
            \Expl}
    \In}
    else
    \In{
        Swap(A[i], A[j]) // Swap root element with (larger) child
        i <- j
    \In}
\In}        
\Code}

\Code{
IndexOfLargestk
if 2*i < n and A[2*i] < A[2*i+1] then
    \Expl{  The left child of A[i] is A[2*i] and the right child (if there is
            a right child) is A[2*i+1]; set j to the index of the larger child.
    \Expl}
\In{
    j <- 2*i+1
\In}
else
\In{
    j <- 2*i
\In}
\Code}

\Code{
SortHeap
// Sort heap
while n>1
    \Expl{  A[1] always has the largest value not yet processed in the 
            sorting phase. A[n] is the last array element in the heap-ordered
            array that is not yet sorted. Repeatedly swap these two values, 
            so that the largest element is now in the last place, decrement n 
            and re-establish the heap condition for the remaining heap (which
            now has one less element). Repeat this procedure until n=1, that 
            is, only one node remains.  
    \Expl}
\In{
    Swap(A[n], A[1])
    n <- n-1    
    DownHeap(A, 1, n) \Ref DownHeap1
        \Expl{  Now that the root node has been swapped to the end, A[1] may 
                no longer be the largest element in the (reduced size) heap.
                Use the DownHeap operation to restore the heap condition. 
        \Expl}
\In}

\Note{  This is very similar to DownHeapk.
\Note}
\Code
DownHeap1
// DownHeap(A, 1, n)
i <- 1
    \Expl{  Set index i to the root of the subtree that we are now going to 
            examine. 
    \Expl}
heap <- False // 'heap' is a flag
while not (IsLeaf(A[i]) or heap) do 
    \Expl{  Traverse down the heap until the current node A[i] is a leaf. 
            We also terminate the loop if the children of A[i] are in the 
            correct order relative to the parent, since we know that subtrees
            lower down already meet the heap condition. We use the heap flag
            to test the heap condition.  
    \Expl}
\In{        
    j <- IndexOfLargestChild(A, i, n) \Ref IndexOfLargest0 
        \Expl{  Find the larger of the two children of the node. 
        \Expl}
    if A[i] >= A[j] then    // Parent is larger than the largest child
    \In{
        heap <- True 
            \Expl{  The heap condition is satisfied, that is, the root is 
                    larger than both children, so we exit from the while loop.
            \Expl}
    \In}
    else
    \In{
        Swap(A[i], A[j]) // Swap root element with (larger) child
        i <- j
    \In}
\In}        
\Code}

\Note{  Same as IndexOfLargestk (could possible reuse that; it is duplicated 
        here because it might make linking with animation easier if each code 
        expansion is used from a single place).
\Note}
\Code{
IndexOfLargest0
if 2*i < n and A[2*i] < A[2*i+1] then
    \Expl{  The left child of A[i] is A[2*i] and the right child (if there is
            a right child) is A[2*i+1]; set j to the index of the larger child.
    \Expl}
\In{
    j <- 2*i+1
\In}
else
\In{
    j <- 2*i
\In}
\Code}

\Note{
Implementation in C omitted for brevity
\Note}

\end{verbatim}

\end{document}
